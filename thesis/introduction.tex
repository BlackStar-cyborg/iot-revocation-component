\chapter{Introduction}

\section{Motivation}

There are daily more and more Internet of Things, IoT, devices connected to the internet, with the need to gather and
process massive amounts of real-time information, especially with 5th Generation networking, which allows for extensive
information exchange.

Enhanced connectivity and adoption of IoT trigger cyber attacks, which are increasingly sophisticated and affect
considerable amount of IoT-related infrastructures, raising security concerns with consumers, as well as businesses.
This stresses the importance of appropriate IoT security management and enhancement of IoT life-cycle management.
Considering the heterogeneity of IoT devices, the dynamism of the security landscape and number of IoT stakeholders make
these tasks quite challenging, especially considering that a single insecure update can put a whole IoT system at
risk.

As emphasized by the European Union Agency for Cybersecurity, ENISA, Cyber Security Act, CSA, the management of IoT
infrastructures encompassing the entire life-cycle of products, as well as the continuous certification, are fundamental
tools to guarantee a high level of security. \cite{EU-cybersecurity-act}
Also as pointed out by the Network and Information Security, NIS2, directive a pragmatic security framework must
stimulate active collaboration between the IoT Stakeholders. \cite{EU-nis2-directive}

Providing access to cybersecurity information is central for realization of a homogeneous perspective on cybersecurity.
CSA and NIS promote strategic cooperation among stakeholders to support and facilitate information sharing, leading to
an approach that helps respond to large-scale incident by creating more effective synergies against cybersecurity
vulnerabilities.

\section{Description of Work}

This thesis will develop a service to support security information sharing between IoT stakeholders to support
continuous security assessment throughout the IoT device life-cycle.
We will consider the use of Distributed Ledger Technology, DLT, as a possible approach to facilitate a trustworthy and
transparent platform for sharing cybersecurity information without a trusted third-party. \cite{neisse2017blockchain}
It is important to integrate the presence of several entities with different responsibilities and roles in sharing
cybersecurity knowledge. So while performing security monitoring activities, the user, i.e., a device, may detect
vulnerabilities that will be shared with the manufacturer for further investigation, prompting for mitigation and or
resolutions. This requires the device to be re-configurable throughout its life-cycle according to the changing threat
landscape and to the device manufacturers publishing of secure updates, i.e., patches, and device profiles.

Established approaches of secure IoT deployment and bootstrapping have significant challenges. Using pre-shared
credentials for every device is the simplest approach, but it prevents the identification of specific devices
and the verification, whether the device is corrected to the correct network. \cite{trusted-iot-draft} This thesis will
develop a bootstrapping service to provide a lightweight bootstrapping protocol, supporting different authentication
methods, depending on the characteristics of the device and providing key management.

The infrastructure of the developed bootstrapping mechanism will enable the inventorying of IoT devices. Said
infrastructure will keep track of all embedded IoT devices and their respective security levels.
To ensure security throughout the life-cycle of a device, an update/patching mechanism will be developed, where
manufacturers and software providers will provide fixes to a security issue after an attack or vulnerability detection.

As most updating proposals are based on centralized models using e.g., client-server architectures, this thesis will
design a scalable and secure approach for disseminating software updates in scenarios with selected IoT devices.
The design shall entail decentralization, robustness and efficiency, bringing the upgrading functionality closer to the
end devices. Blockchain based technologies will be leveraged by providing a transparent ledger to manage different
versions of software elements composing and IoT device or system and share relevant security aspects, such as
vulnerabilities or device information.
% TODO: (aver) Maybe remove this part, I'm unsure how far interoperability will go with Interledger/Bifröst.
As interoperability is crucial, this thesis will analyze the use of Bifröst \cite{scheid2019bifrost} and Interledger
\cite{siris2019interledger} approaches to interconnect different blockchain implementations.

Finally this thesis will consider mitigation for IoT devices using Threat Manufacturer Usage Description, MUD,
proposed by NIST \cite{dodson2021securing}, which provides a flexible and dynamic way to alert about new threats and
mitigation to apply before and update of patch is released. Threat MUD is intended as a complement to MUD file,
dynamically reconfiguring a device in case of detection of a vulnerability.

% TODO: (aver) finish up after main parts are done/implemented
\section{Thesis Outline}

This thesis has two main emphasis, Inventorying and Secure Firmware Updating.

IoT devices need to able to be uniquely identified \dots
