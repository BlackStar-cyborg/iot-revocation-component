\chapter*{Glossary}
\addcontentsline{toc}{chapter}{Glossary}
\markboth{GLOSSARY}{}

% TODO: (aver) Ask Eryk if good as is
% https://csrc.nist.gov/glossary

\begin{description}
	% \item[Authentication]
	% \item[Accounting]
	% \item[Authorization] Authorization is the decision whether an entity is allowed to perform a particular action
	% 	or not, e.g. whether a user is allowed to attach to a network or not.
	\item[Trust Model] In the trust model the issuer issues credential to a holder while the holder can prove identity by
		showing the credential to a verifier.
	\item [Manufacturer Usage Description] A component-based architecture specified in Request for Comments (RFC)
	      8520 that is designed to provide a means for end devices to signal to the network what sort of access and
	      network functionality they require to properly function.
	\item [Cloud Computing] Cloud computing is the on-demand availability of computer system resources, especially
	      data storage and computing power, without direct active management by the user.
	\item [Fog Computing] As an extension of Cloud computing, Fog Computing brings the computation closer to IoT
	      Edge devices.
	\item [Edge Computing] Edge computing is the placement of storage and computing resources closer to source, where
	      the data is generated.
	\item [Trusted Execution Zone]
	\item [Line-Replaceable Unit] modular component of airplane, designed to be replaced quickly
\end{description}
