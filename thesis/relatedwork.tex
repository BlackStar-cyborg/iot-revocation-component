\chapter{Related Work}

% compares, contrasts, synthesizes, and provides introspection about the available knowledge for a given topic or field

\section{CERTIFY}

This thesis is carried out in conjunction with the CERTIFY project.

The National Institute for Standard and Technology has a few ongoing projects and white papers on security related
mitigation methods for IoT devices.

% \section{NIST}
% \subsection{Trusted IoT Device Network-Layer Onboarding and Lifecycle Management \cite{trusted-iot-draft}}
% \subsection{Securing Small-Business and Home Internet of Things (IoT) Devices: Mitigating Network-Based Attacks Using
% 	Manufacturer Usage Description (MUD) \cite{dodson2021securing}}

\section{DLT-based Asset-Tracking}

Neisse et al. (2017) analyzed how blockchain-based approaches might be used for data accountability and provenance
tracking under the then recently released GDPR legislation, highlighting challenges of scalability and considering
sharding as a method to address it. \cite{neisse2017blockchain} Further they also mentioned issues of clonability of
the tracked assets, which we can also correlate to the physical assets that are tracked inside blockchain.

\section{Device fingerprinting}
% possibly merge this into upper section

In order to be able to track IoT nodes in a blockchain, they need to be uniquely identifiable, in our case even in a
distributed manner, using Distributed Identifiers, DIDs. Methods of creating identifiers that are unique to devices
exist, such as SRAM-Based PUF Readouts \cite{vinagrero2023sram}.

\section{Cybersecurity of IoT Devices}

In order to maintain participation rights for only valid users/clients, Manufacturer Usage Descriptions, MUDs, are
getting more and more relevant, as also the National Institute for Standards and Technology, NIST, have been considering
their use cases. \cite{dodson2021securing}
